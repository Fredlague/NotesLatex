%--------------------------INITIALISATION DU DOCUMENT---------------------%
%											%														%											%
%                        >>>> Ne pas modifier cette partie <<<<			%
																																					
\documentclass[letterpaper,12pt,oneside,final]{book}
\input{packages}


\begin{document}
%--------------------------------------------------------------------------------------%

%--------------------------PAGE DE COUVERTURE------------------------------%

% A REMPLIR PAR L'ETUDIANT: 

\newcommand\monPrenom{Frédéric}		%PRENOM
\newcommand\monNom{Laguë}			%NOM
\newcommand\monMatricule{1986131}	%MATRICULE
\newcommand\monGroupe{01}		%GROUPE

%------------------------ Ne pas modifier la ligne suivante --------------%
%\newgeometry{tmargin=2.0cm, bmargin=2.0cm, lmargin=2.25cm, rmargin=2.25cm, headsep=1.0cm}
\newgeometry{top=2cm}
\definecolor{gris1}{gray}{0.75}

\newcommand{\encadre}[1]{
\setlength\fboxsep{5mm}\setlength\fboxrule{1pt}
\begin{center}
\fcolorbox{black}{gris1}{
\begin{minipage}{0.94\textwidth}{#1}\end{minipage}}
\end{center}}

% encadre blanc
\newcommand{\boite}[1]{
\setlength\fboxsep{5mm}\setlength\fboxrule{1pt}
\begin{center}
\fcolorbox{black}{white}{
\begin{minipage}{0.5\textwidth}{#1}\end{minipage}}
\end{center}}


%\begin{document}

\thispagestyle{empty}

{
\centering

\encadre{
\begin{center}
\bf
{\Large \scshape 
Polytechnique Montr\'eal
\\
D\'epartement de Math\'ematiques et de G\'enie Industriel
}
\\
{\Huge
\

MTH3400 - Analyse math\'ematique pour ing\'enieurs
\\
Automne 2021

\

Devoir 2

}
\end{center}
}

\vfill

\fcolorbox{black}{white}{
\begin{minipage}{0.94\linewidth}

\vspace{5mm}

{\bf \Large Nom: }\monNom \hspace{20mm} {\bf \Large Pr\'enom: }\monPrenom%\rule[-1mm]{56mm}{0.6pt}

\vspace{8mm}

{\bf \Large Matricule: }\monMatricule \hspace{20mm} {\bf \Large Section: }\monGroupe

%\vspace{8mm}
%
%{\bf \Large Signature: }\rule[-1mm]{126mm}{0.6pt}
%
%

\end{minipage}}

\vspace{10mm}


{
\renewcommand{\arraystretch}{1.5}
\begin{center}
\begin{tabular}{|c|c|c|c||c|} \hline
{\bf \Large Q1}		& {\bf \Large Q2}	& {\bf \Large Q3}& 	{\bf \Large Q4} 	& {\bf \Large Total} \\ \hline
\hspace{20mm}	{\Huge \strut}	& \hspace{20mm}	& \hspace{20mm}	& \hspace{20mm}  & \hspace{20mm} \\ 
\hspace{20mm}	{\Huge \strut}	& \hspace{20mm}	& \hspace{20mm}	&  \hspace{20mm}  & \hspace{20mm} \\ \hline
\end{tabular}
\end{center}
}


\vfill

}

\restoregeometry
%\end{document}
%-------------------------------------------------------------------------%


%========================= Début des réponses ============================%


%-----------------------------QUESTION 1 ----------------------------------%
\section*{Question 1}

Soient \( E_{1,},E_{2},\ldots,E_{n} \) une collection dénombrable de sous-ensembles de mesures nulle de \( \mathbb{R}^{n} \). \\ 
On peut construire une collection \( F_{1},F_{2},\ldots,F_{n} \) de sous-ensembles disjoints tels que 
\[ \cdot \hspace{-11.90pt}\bigcup_{k=1}^{n} F_{n} = \bigcup_{k=1}^{n} E_{n}, \]
en prenant
\[ F_{1} = E_{1} \]
\[ F_{2} = E_{2} \textbackslash E_{1}   \]
\[ F_{3} = E_{3} \textbackslash (E_{1}\bigcup E_{2}) \]
\[ . \]
\[ . \]
\[ . \]
\[ F_{n} = E_{n} \textbackslash (E_{1} \bigcup E_{2}\bigcup \ldots \bigcup E_{n-1} )\]

De la définition d'une mesure, la mesure d'une union dénombrable d'ensembles disjoints est
\[ \mu\left(\hspace{11pt}\cdot\hspace{-11.9pt}\bigcup_{n}^{\infty }F_{n}\right) = \sum_{n=1}^{\infty} \mu \left(F_{n}\right) \]

Or, par construction, chaque \( F_{n}  \) est inclus dans \( E_{n} \) 
\[ F_{n} \subset E_{n} \]
\[ \implies \mu (F_{n}) \leq \mu (E_{n}) \]
 % A REMPLIR PAR L'ETUDIANT:
On a donc 
\[ \mu\left( \bigcup_{k=1}^{n} E_{k} \right) = \mu\left(\hspace{11pt}\cdot \hspace{-11.9pt}\bigcup_{k=1}^{n} F_{k}\;\right) = \sum_{k=1}^{n} \mu\left( F_{k} \right) \leq \sum_{k=1}^{N} \mu\left( E_{n} \right) = 0 \]


%-----------------------------QUESTION 2 ----------------------------------%
\newpage
\section*{Question 2}

\begin{enumerate}[a)]

\item % a)
On peut écrire la fonction valeur absolue comme 
\[ f(x) = \lvert x \rvert  = \begin{cases}
	x, &\text{ si } x \geq  0 \\ 
	-x, &\text{ si } x<0
\end{cases} \]
On a \( f \in L^{1}(\mathbb{R} ) \). La fonction \( f \) définit donc un fonctionnelle \( l_{f}\in \mathcal{D}(\mathbb{R} )^{*} \) telle que, \( \forall \varphi \in \mathcal{D}(\mathbb{R} ) \)
\[ l_{f}(\varphi) = \int_{\mathbb{R} }f(x)\cdot \varphi (x) \text{d}x\; \]
On peut choisir \( \varphi  \) à support compact, telle que \( \overline{\text{supp}(\varphi)} = [-M,M] \), pour une constante \( M \).
On a donc la dérivée au sens large de \( f \); 
\begin{equation*}
\begin{split}
	\partial_{x}l_{f}(\varphi) &= - \int_{\mathbb{R} }f(x) \frac{\text{d}\varphi (x)}{\text{d}x} \text{d}x\;\\ 
& = - \int_{-M}^{M}\text{d}x f(x) \frac{\text{d}\varphi (x)}{\text{d}x}\;\\ 
& = -\int_{-M}^{0}\text{d}x\;f(x)\frac{\text{d}(\varphi(x))}{\text{d}x} - \int_{0}^{M}\text{d}x\;f(x) \frac{\text{d}\varphi (x)}{\text{d}x} \\ 
& = -\int_{-M}^{0}\text{d}x\;f(x)\frac{\text{d}\varphi (x)}{\text{d}x} -\int_{0}^{M}\text{d}x\;f(x)\frac{\text{d}(\varphi (x))}{\text{d}x}\\ 
& = -\lim_{b \to 0^{-}}\Bigg(\Bigl[\lvert x \rvert \varphi (x)\Bigr]_{-M}^{b} - \int_{-M}^{b}\text{d}x\;\frac{\text{d}f(x)}{\text{d}x} \varphi(x)\Bigg)\\
& - \lim_{a \to 0^{+}} \Biggl(\Bigl[f(x)\varphi (x)\Bigr]_{a}^{M}-\int_{a}^{M}\text{d}x\;\frac{\text{d}f(x)}{\text{d}x} \varphi (x)\Biggr)\\
& = -\lim_{b \to 0^{-}} \Biggl(\Bigl[-b\cdot \varphi (b)-(-M)\cdot \varphi(-M)\Bigr]-\int_{-M}^{b}\text{d}x\;\frac{\text{d}f(x)}{\text{d}x}\varphi(x)\Biggr)\\
& -\lim_{a \to 0^{+}}\Biggl(\Bigl[M\cdot \varphi (M)-a\cdot \varphi (a)\Bigr]-\int_{a}^{M}\text{d}x\;\frac{\text{d}f(x)}{\text{d}x} \varphi (x)\Biggr) 
\end{split}
\end{equation*}
Avec \( \varphi(-M)=\varphi(M) =0 \), on obtient 

\begin{equation*}
\begin{split}
	 \partial_{x}l_{f}(\varphi) & = \lim_{b \to 0^{-}}  \int_{-M}^{b}\text{d}x\;\frac{\text{d}f(x)}{\text{d}x} \varphi(x) + \lim_{a \to 0^{+}} \int_{a}^{M}\text{d}x\; \frac{\text{d}f(x)}{\text{d}x} \varphi(x)\\
& = -\lim_{b \to 0^{-}} \int_{-M}^{b}\text{d}x\;\varphi (x) + \lim_{a \to 0^{+}} \int_{a}^{M}\text{d}x\;\varphi (x)
\end{split}
\end{equation*}
En posant
\[ g(x) = \text{sign}(x) = \begin{cases}
	1, & \text{ si } \geq 0\\ 
	-1, & \text{ si } x < 0
\end{cases} \]
On a donc
\begin{equation*}
\begin{split}
	\partial_{x}l_{f}(\varphi) & = \int_{-M}^{M}\text{d}x\;g(x)\varphi(x) 
\end{split}
\end{equation*}

\[ \implies \partial_{x}l_{f} = \partial_{x}l_{g} \]
Puisque l'égalité est respectée pour toute fonction test \( \varphi \), la dérivée au sens large de \(  f(x) = \lvert x \rvert  \) est donc \( g(x) = \text{sign}(x)\)
 % A REMPLIR PAR L'ETUDIANT:

\item % b)

	\begin{enumerate}[(i)]
		
% A REMPLIR PAR L'ETUDIANT:
	
	\item % (ii) 
Soit \( \omega \in D(\mathbb{R} )  \) telle que \( \int_{\mathbb{R} }\text{d}x\;\omega(x) = 1\). Alors, on peut définir 
\[ \psi(t) = \varphi(t) - \omega(t) \int_{\mathbb{R} }\text{d}s\;\varphi(s) \]

D'abord, on calcule la dérivée de \( \psi \). 
\begin{equation*}
\begin{split}
	\frac{\text{d}\psi}{\text{d}t} & = \frac{\text{d}}{\text{d}t} \left(\varphi(t) - \omega(t) \int_{\mathbb{R} }\text{d}s\;\varphi(s)\right) \\ 
& = \varphi '(t) - \omega'(t)\int_{\mathbb{R} } \varphi(s)\text{d}s = 0\\ 
& \implies \psi \in C^{\infty }
\end{split}
\end{equation*}

Ainsi, \( \forall \varphi \in D(\mathbb{R}) \),
\begin{equation*}
\begin{split}
	\int_{\mathbb{R} }\text{d}t\;\psi(t) & = \int_{\mathbb{R}}\left(\varphi(t) - \omega(t)\int_{\mathbb{R}} \varphi(s) \text{d}s\right)\text{d}t \\ 
& = \int_{\mathbb{R} } \varphi (t) dt - \int_{\mathbb{R} }\left( \omega(t)\int_{\mathbb{R}}\varphi(s)\text{d}s   \right) \text{d}t\\ 
& = 1- \left( 1\cdot 1 \right) \\ 
& = 0
\end{split}
\end{equation*}
Selon le théorème fondamental du calcul intégral, 
\[ \frac{\text{d}}{\text{d}t} \int_{-\infty }^{t}\text{d}x\;f(x) = f(t)  \]
Donc, 
\[ \varphi '(t) = \frac{\text{d}}{\text{d}t} \left( \varphi(r) - \omega(r)\int_{\mathbb{R} }\text{d}r\; \right)  = \varphi(t) -\omega(t) \int_{\mathbb{R}}\varphi(s)\text{d}s\;   =\psi(t)  \]

\item 
On a
\begin{equation*}
\begin{split}
l_{f}(\varphi) & = \int_{\mathbb{R} }\text{d}x\; f(x) \varphi(x)  \\ 
& = \int_{\mathbb{R} }\text{d}x\; f(x)\left( \psi(t) +\omega(x) \int_{\mathbb{R} }\varphi(s)\text{d}s\; \right) \\ 
& = \int_{\mathbb{R} }\text{d}x\; f(x) \psi(x) + \int_{\mathbb{R} }\text{d}\; f(x)\omega(x)\int_{\mathbb{R} }\varphi(s)\text{d}s\;  \\ 
& = \int_{\mathbb{R} }\text{d}\; f(x) \varphi '(x) + \int_{\mathbb{R} }\text{d}x\;f(x)\omega(x)\\ 
& = 0 + \int_{\mathbb{R} }\text{d}x\; C f(x)\omega(x)\\ 
& = l_{C}(x)
\end{split}
\end{equation*} 
	
\end{enumerate}

\end{enumerate}



%-----------------------------QUESTION 3 ----------------------------------%
\newpage
\section*{Question 3}

\[ a^{\dagger} \ket{0} = \ket{1}\]
\[ \ket{n} = \frac{1}{\sqrt{n!}} a^{\dagger\,n} \ket{0} \]
\[ a^{\dagger}_{k} \ket{0_{1},0_{2},\ldots,0_{k-1},0_{k},\ldots} =\ket{0_{1},0_{2},\ldots,0_{k-1},1_{k},\ldots}\]
\[ \ket{n_{1},n_{2},n_{3},\ldots} = \frac{1}{\sqrt{n_{1}!}} a^{\dagger\,n_{1}}\frac{1}{\sqrt{n_{2}!}} a^{\dagger\,n_{2}}\frac{1}{\sqrt{n_{3}!}} a^{\dagger\,n_{3}} \ket{0,0,0,\ldots}\]
\[n_{k} = a_{k}^{\dagger}a_{k} \]
\[ n_{k} \ket{n_{1}n_{2},\ldots, n_{k},\ldots} = n_{k} \ket{n_{1}n_{2},\ldots, n_{k},\ldots}\]
\[ N_{tot} = \sum_{k} a_{k}^{\dagger}a_{k} \]

Soient \( \Omega \subset \mathbb{R}^{3}\) un ouvert borné. On veut montrer que 
\[ \int_{\Omega}\text{d}x\;\vb{u}\cdot (\nabla \times \vb{v}) = \int_{\Omega}\text{d}x\;(\nabla \times \vb{u}) \cdot  \vb{v} ,\]
\( \forall \vb{u},\vb{v} \in H_{0}^{1}(\Omega) \)

On définit la fonction \( f:\, \mathbb{R}^{3} \times \mathbb{R}^{3} \rightarrow  \mathbb{R} \) comme 
\[ f(\vb{u},\vb{v}) =   \int_{\Omega}\text{d}x\;\vb{u}\cdot (\nabla \times \vb{v}) - \int_{\Omega}\text{d}x\;(\nabla \times \vb{u}) \cdot  \vb{v} \]

On montre d'abord que \( f \) est bilinéaire:

i) 
\begin{equation*}
\begin{split}
	f(\alpha_{1}\vb{u}_{1}+\alpha_{2}\vb{u_{2}}, v) & = \int_{\Omega}\text{d}x\; (\alpha_{1}\vb{u}_{1}+\alpha_{2}\vb{v}_{2}) \cdot (\nabla \times \vb{v}) - \int_{\Omega }\text{d}x\;(\nabla \times \left[\alpha\vb{u}_{1}+\alpha_{2}\vb{u}_{2}\right])\cdot \vb{v}\\ 
& = \int_{\Omega}\text{d}x\;\alpha_{1}\vb{u}_{1}\cdot (\nabla \times \vb{v} ) + \alpha_{2}\vb{u}_{2} \cdot (\nabla \times \vb{v} ) \\
& \;\;- \int_{\Omega }\text{d}x\;\alpha_{1}(\nabla \times \vb{u}_{1} ) \cdot \vb{v} + \alpha_{2}(\nabla \times \vb{u}_{2} )\cdot \vb{v}  \\ 
& = \int_{\Omega }\text{d}x\;\alpha_{1}\vb{u}_{1}\cdot (\nabla \times \vb{v}) + \int_{\Omega }\text{d}x\;\alpha_{2}\vb{u}_{2}\cdot \nabla \times \vb{v} \\ 
& \;\;- \int_{\Omega }\text{d}x\;\alpha_{1}(\nabla \times \vb{u}_{1} ) \cdot \vb{v} - \int_{\Omega }\text{d}x\;\alpha_{2}(\nabla \times \vb{u}_{2} )\cdot \vb{v}\\ 
& = \alpha_{1}\int_{\Omega }\text{d}x\;\vb{u}_{1}\cdot (\nabla \times \vb{v}) - \alpha_{2}\int_{\Omega }\text{d}x\;\vb{u}_{1}\cdot \nabla \times \vb{v} \\
& \;\;- \alpha_{1}\int_{\Omega }\text{d}x\;(\nabla \times \vb{u}_{1} ) \cdot \vb{v} - \alpha_{2}\int_{\Omega }\text{d}x\;(\nabla \times \vb{u}_{2} )\cdot \vb{v}\\
& = \alpha_{1}\left( \int_{\Omega }\text{d}x\;\vb{u}_{1}\cdot (\nabla \times \vb{v}) - \int_{\Omega }\text{d}x\;(\nabla \times \vb{u}_{1} ) \cdot \vb{v} \right) \\ 
& \;\;+ \alpha_{2}\left( \int_{\Omega }\text{d}x\;\vb{u}_{2}\cdot (\nabla \times \vb{v}) - \int_{\Omega }\text{d}x\;(\nabla \times \vb{u}_{2} ) \cdot \vb{v} \right) \\ 
& = \alpha_{1}f(\vb{u}_{1},\vb{v}_{1}) + \alpha_{2}f(\vb{u}_{2},\vb{v})
\end{split}
\end{equation*}
ii)
\begin{equation*}
\begin{split}
	f(\vb{u},\alpha{1}\vb{v}_{1}+\alpha_{2}\vb{v}_{2}) &= \int_{\Omega }\text{d}x\; \vb{u}\cdot \left(\nabla \times \left[ \alpha_{1}\vb{v}_{1}+\alpha_{2}\vb{v}_{2} \right] \right) - 
\int_{\Omega }\text{d}x\;\left( \nabla \times \vb{u}  \right) \cdot (\alpha_{1}\vb{v}_{1}+\alpha_{2}) \\ 
& = \int_{\Omega}\text{d}x\; \vb{u}\cdot (\alpha_{1} \left[ \nabla \times \vb{v}_{1} \right] + \alpha_{2}\left[ \nabla \times \vb{v}_{2}  \right])\\ 
&\;\; - \int_{\Omega}\text{d}\; (\nabla \times \vb{u} ) \cdot \alpha_{1}\vb{v}_{1} + (\nabla \times \vb{u} ) \cdot \alpha_{2}\vb{v}_{2} \\ 
& = \alpha_{1}\int_{\Omega}\text{d}\; \vb{u}\cdot (\nabla \times \vb{v}_{1} ) + \alpha_{2}\int_{\Omega}\text{d}\; \vb{u}\cdot (\nabla \times \vb{v}_{2} ) \\ 
&\;\; -\alpha_{1}\int_{\Omega}\text{d}\; (\nabla \times \vb{u} )\cdot \vb{v}_{1} -\alpha_{2}\int_{\Omega}\text{d}\; (\nabla \times \vb{u} )\cdot  \vb{v}_{2} \\ 
& = \alpha_{1}(\int_{\Omega}\text{d}\; \vb{u}\cdot (\nabla \times \vb{v}_{1} ) - \int_{\Omega}\text{d}\; (\nabla \times \vb{u} )\cdot \vb{v}_{1} ) \\ 
& \;\; + \alpha_{2}(\int_{\Omega}\text{d}\; \vb{u}\cdot (\nabla \times \vb{v}_{2} ) - \int_{\Omega}\text{d}\; (\nabla \times \vb{u} )\cdot \vb{v}_{2} ) \\ 
& = \alpha_{1}f(\vb{u},\vb{v}_{1}) + \alpha_{2}f(\vb{u},\vb{v}_{2})
\end{split}
\end{equation*}
La fonction \( f \) est donc bien bilinéaire. \\ 
Ensuite, on montre que \( f \) est bornée. \\ 

On a
\begin{equation*}
\begin{split}
	 \lvert f(\vb{u},\vb{v}) \rvert & \leq \int_{\Omega}\text{d}x\; \lvert \vb{u}\cdot (\nabla \times \vb{v} ) \rvert + \int_{\Omega}\text{d}x\; \lvert (\nabla \times \vb{u} ) \cdot \vb{v}\rvert    \\ 
	& \leq \lvert \lvert  \vb{u} \rvert \rvert_{L^{2}(\Omega)^{3}} \; \lvert \lvert  \nabla \times \vb{v} \rvert \rvert_{{L^{2}(\Omega )^{3}}} + \lvert \lvert  \nabla \times \vb{u}  \rvert \rvert_{L^{2}(\Omega )^{3}} \lvert \lvert  \vb{v} \rvert \rvert_{L^{2}(\Omega )^{3}} \\ 
\end{split}
\end{equation*}
, où l'on a utilisé l'identité de Cauchy-Schwarz. \\
Puisque les dérivées partielles sont continues, l'opérateur \( \nabla \times : H_{0}^{1}(\Omega)^{3} \rightarrow H_{0}^{0}(\Omega)^{3}\), en étant une combinaison linéaire des dérivées partielles est continue, donc borné. \\ Ainsi, il existe deux constantes 
\( M_{\vb{v}} \text{ et } M_{\vb{u}} \) telles que 
\[ \lvert \lvert  \nabla \times \vb{v}  \rvert \rvert_{H^{1}_{0}(\Omega)^{3}} \leq M_{\vb{v}} \lvert \lvert  \vb{v} \rvert \rvert_{H_{0}^{1}(\Omega)^{3}}  \] 
et 
\[ \lvert \lvert  \nabla \times \vb{u}  \rvert \rvert_{H^{0}_{0}(\Omega)^{3}} \leq M_{\vb{u}} \lvert \lvert  \vb{u} \rvert \rvert_{H_{0}^{1}(\Omega)^{3}}  \] 

De plus, on a par définition 
\[ \lvert \lvert  \vb{u} \rvert \rvert_{H_{0}^{1}(\Omega )^{3}} \geq \lvert \lvert  \vb{u} \rvert \rvert_{L^{2}(\Omega)^{3}} \text{      et      } \lvert \lvert  \vb{u} \rvert \rvert_{H_{0}^{1}(\Omega )^{3}} \geq \lvert \lvert  \vb{v} \rvert \rvert_{H_{0}^{1}(\Omega ^{3})} \]

On peut donc réécrire 
\[ \lvert f(\vb{\vb{v}}) \rvert \leq \lvert \lvert  \vb{u} \rvert \rvert_{H_{0}^{1}(\Omega )^{3}} M_{\vb{v}} \lvert \lvert  \vb{v} \rvert \rvert_{H_{0}^{1}(\Omega)^{3}} + \lvert f(\vb{\vb{v}}) \rvert \leq \lvert \lvert  \vb{v} \rvert \rvert_{H_{0}^{1}(\Omega )^{3}} M_{\vb{u}} \lvert \lvert  \vb{u} \rvert \rvert_{H_{0}^{1}(\Omega)^{3}}   \]
\[ \implies \lvert f(\vb{u},\vb{v}) \rvert \leq (M_{\vb{v}}+M_{\vb{u}}) \lvert \lvert  \vb{u} \rvert \rvert_{H_{0}^{1}(\Omega )^{3}} \lvert \lvert  \vb{v} \rvert \rvert_{H_{0}^{1}(\Omega )^{3}} \]
On a donc que \( f \) est une fonction bilinéaire et bornée entre deux espaces vectoriels normés, alors \( f \) est continue sur \( H_{0}^{1}(\Omega)^{3} \)
\\ Ensuite, il est donnée que \( \forall \phi, \psi \in C^{1}_{0}(\Omega )^{3} \), on a 
\[ f(\phi,\psi) = 0 \]

Or, puisque \( H_{0}^{1} \) est la complétude de \( C_{0}H^{1} \), on a \( C_{0}^{1} \subset H_{0}^{1} \). 
De plus, \( C_{0}^{1} \) est dense. \\
Ainsi, puisque \( f = 0  \) sur un sous-ensemble dense et \( f \) est continue sur \( H_{0}^{1}(\Omega )^{3} \), alors 

\[ f(\vb{u},\vb{v}) = 0,\;\; \forall \vb{v},\vb{u} \in H_{0}^{1}(\Omega )^{3} \]
 % A REMPLIR PAR L'ETUDIANT:



%-----------------------------QUESTION 4 ----------------------------------%



\end{document}
