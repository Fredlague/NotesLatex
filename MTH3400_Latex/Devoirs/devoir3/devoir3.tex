%--------------------------INITIALISATION DU DOCUMENT---------------------%
%											%														%											%
%                        >>>> Ne pas modifier cette partie <<<<			%
																																					
\documentclass[letterpaper,12pt,oneside,final]{book}



%%
%%  Version: 2014-10-28
%%
%%  Accepte les caractères accentués dans le document (UTF-8).
\usepackage[utf8]{inputenc}
%%
\usepackage{amsfonts}
%% Support pour l'anglais et le français (français par défaut).
%\usepackage[cyr]{aeguill}
\usepackage{lmodern}      % Police de caractères plus complète et généralement indistinguable visuellement de la police standard de LaTeX (Computer Modern).
\usepackage[T1]{fontenc}  % Bon encodage des caractères pour qu'Acrobat Reader reconnaisse les accents et les ligatures telles que ffi.
\usepackage[english,frenchb]{babel} % le langage par défaut est le dernier de la liste, c'est-à-dire français
%%
%% Charge le module d'affichage graphique.
\usepackage{graphicx}
\usepackage{epstopdf}  % Permet d'utiliser des .eps avec pdfLaTeX.
%%
%% Recherche des images dans les répertoires.
\graphicspath{{./images/}{./dia/}{./gnuplot/}}
%%
%% Un float peut apparaître seulement après sa définition, jamais avant.
\usepackage{flafter,placeins}
%%
%% Utilisation de natbib pour les citations et la bibliographie.
\usepackage{natbib}
%%
%% Autres packages.
\usepackage{amsmath,color,soulutf8,longtable,colortbl,setspace,ifthen,xspace,url,pdflscape,tikz,pgfplots}
%%
%% Support des acronymes.
\usepackage[nolist]{acronym}
\onehalfspacing                % Interligne 1.5.
%%
%% Définition d'un style de page avec seulement le numéro de page à
%% droite. On s'assure aussi que le style de page par défaut soit
%% d'afficher le numéro de page en haut à droite.
\usepackage{fancyhdr}
\fancypagestyle{pagenumber}{\fancyhf{}\fancyhead[R]{\thepage}}
\renewcommand\headrulewidth{0pt}
\makeatletter
\let\ps@plain=\ps@pagenumber
\makeatother
%%
%% Module qui permet la création des bookmarks dans un fichier PDF.
%\usepackage[dvipdfm]{hyperref}
\usepackage{hyperref}
\usepackage{caption}  % Hyperlien vers la figure plutôt que son titre.

\usepackage{esint}
\usepackage{geometry}
\usepackage{enumerate}


\usepackage[]{physics}


\begin{document}
%--------------------------------------------------------------------------------------%

%--------------------------PAGE DE COUVERTURE------------------------------%

% A REMPLIR PAR L'ETUDIANT: 

\newcommand\monPrenom{Frédéric}		%PRENOM
\newcommand\monNom{Lague}			%NOM
\newcommand\monMatricule{186131}	%MATRICULE
\newcommand\monGroupe{01}		%GROUPE

%------------------------ Ne pas modifier la ligne suivante --------------%
%\newgeometry{tmargin=2.0cm, bmargin=2.0cm, lmargin=2.25cm, rmargin=2.25cm, headsep=1.0cm}
\newgeometry{top=2cm}
\definecolor{gris1}{gray}{0.75}

\newcommand{\encadre}[1]{
\setlength\fboxsep{5mm}\setlength\fboxrule{1pt}
\begin{center}
\fcolorbox{black}{gris1}{
\begin{minipage}{0.94\textwidth}{#1}\end{minipage}}
\end{center}}

% encadre blanc
\newcommand{\boite}[1]{
\setlength\fboxsep{5mm}\setlength\fboxrule{1pt}
\begin{center}
\fcolorbox{black}{white}{
\begin{minipage}{0.5\textwidth}{#1}\end{minipage}}
\end{center}}


%\begin{document}

\thispagestyle{empty}

{
\centering

\encadre{
\begin{center}
\bf
{\Large \scshape 
Polytechnique Montr\'eal
\\
D\'epartement de Math\'ematiques et de G\'enie Industriel
}
\\
{\Huge
\

MTH3400 - Analyse math\'ematique pour ing\'enieurs
\\
Automne 2021

\

Devoir 3

}
\end{center}
}

\vfill

\fcolorbox{black}{white}{
\begin{minipage}{0.94\linewidth}

\vspace{5mm}

{\bf \Large Nom : }\monNom \hspace{20mm} {\bf \Large Pr\'enom : }\monPrenom%\rule[-1mm]{56mm}{0.6pt}

\vspace{8mm}

{\bf \Large Matricule : }\monMatricule \hspace{20mm} {\bf \Large Groupe : }\monGroupe

%\vspace{8mm}
%
%{\bf \Large Signature: }\rule[-1mm]{126mm}{0.6pt}
%
%

\end{minipage}}

\vspace{10mm}


{
\renewcommand{\arraystretch}{1.5}
\begin{center}
\begin{tabular}{|c|c|c|c||c|} \hline
{\bf \Large Q1}		& {\bf \Large Q2}	& {\bf \Large Q3}& 	{\bf \Large Q4} 	& {\bf \Large Total} \\ \hline
\hspace{20mm}	{\Huge \strut}	& \hspace{20mm}	& \hspace{20mm}	& \hspace{20mm}  & \hspace{20mm} \\ 
\hspace{20mm}	{\Huge \strut}	& \hspace{20mm}	& \hspace{20mm}	&  \hspace{20mm}  & \hspace{20mm} \\ \hline
\end{tabular}
\end{center}
}


\vfill

}

\restoregeometry
%\end{document}
%-------------------------------------------------------------------------%


%========================= Début des réponses ============================%


%-----------------------------QUESTION 1 ----------------------------------%
\section*{Question 1}
Soit l'espace vectoriel
\[
\ell^{1} = \left\{ x\in \mathbb{R}^{\infty} | \lvert x \rvert _{1} < \infty  \right\} 
\]
équipé de la norme additive 
\[
\lvert x \rvert _{1} = \sum_{n=1}^{\infty} | x_{n}|.
\]
Soit le sous-ensemble
\begin{equation}\label{setk}
K = \left\{ x\in \mathbb{R}^{\infty}\;\vert\; \forall n\; \in \; \mathbb{N}_{0}, \;|x_{n}| \leq 2^{-n}\right\}
\end{equation}

 % A REMPLIR PAR L'ETUDIANT:
On veut montrer que $K$ est compact dans \(    \ell^{1}    \), donc que toute suite \(    \left( \vb{x}_{n} \right)_{n}   \) 
possède une sous-suite qui converge dans \(    K   \). C'est-à-dire
\[
\forall \varepsilon > 0,\;\exists\;n_{\varepsilon}\;\text{ tel que } n\geq  n_{\varepsilon}\;\text{implique }
d(\vb{x}_{n},\vb{x}) < \varepsilon 
\]
\[
\iff \lvert\lvert \vb{x_{n}}-\vb{x} \rvert\rvert = \sum_{n=1}^{\infty} \lvert x_{n} \rvert < \varepsilon 
\]


Soit $ \left(\vb{x}_{k}^{(0)} \right)_{k=1}^{\infty} $ une suite d'éléments de $K$. Chaque élément de cette suite est donc 
une suite de $n$ éléments:
\[
\vb{x}_{k}^{(0)}  = \left( (x_{k})_{1},(x_{k})_{2}, (x_{k})_{3},\ldots \right) 
\]
Soit \(  x_{k,j}^{(0)}  \) le $j$-ème élément de la suite \( \left(\vb{x}_{k}^{(0)}\right)  \). La suite
\(
\left( x_{k,j}^{(0)}\right)_{k=1}^{\infty }
\)
est donc bornée entre \( [-\frac{1}{2^{j}}, \frac{1}{2^{j}}] \), car tous les éléments \( \vb{x}_{k}^{(0)}  \) sont dans $K$
et respectent \ref{setk}. De plus, cette suite est dans $\mathbb{R}^{k}$. \\
Ainsi, en choisissant \( j=1 \), la suite \( \left( x_{k,1}^{(0)}\right)_{k=1}^{\infty } \) des premières composantes des \( \vb{x}_{k}^{(0)} \) est une suite dans 
\( [-\frac{1}{2},\frac{1}{2}]\subset \mathbb{R} \).\\
Par exemple, on peut montrer que la suite \( \left( x_{k,j}^{(0)}\right)_{k=1}^{\infty } \) définie comme
\begin{equation} \label{suite1}
    x_{k,j}^{(0)} = \frac{1}{2^{-jk}}
\end{equation}
respecte ces critères.\\
Alors, selon le théorème de Bolzano-Weierstrass, il existe une sous-suite convergente de \(
    \left( x_{k,1}^{(0)}\right)_{k=1}^{\infty } \\
    \) pour certains indices $k$. Le Théorème des intervalles emboités indique que cette limite de cette suite, 
   \( x_{1}^{*} \) ,est dans $[-\frac{1}{2},\frac{1}{2}]$. \\\\
Appelons \( \left(\vb{x}_{k}^{(1)}\right) \), la suite donc les indices $k$ forment la sous-suite convergente mentionnée ci-haut.
De plus, choisissons cette suite telle que 
\[
\vb{x}^{(1)}_{2k} = \vb{x}^{(0)}_{k}
\]
On peut également montrer que la suite définie en \ref{suite1} respecte également ces critères.\\\\
Similairement, prenons cette fois-ci la suite \( \left(\vb{x}_{k,2}^{(1)}\right)_{k=1}^{\infty } \). Puisque 
\[
    \left(\vb{x}_{k}^{(1)}\right)_{k=1}^{\infty } \subset K,
\]
cette suite des deuxièmes composantes de \( \left( \vb{x}_{k}^{(1)}\right) \) est donc dans \( [-\frac{1}{2^{2}},\frac{1}{2^{2}}] \in \mathbb{R} \),
et toujours avec le théorème de Bolzano-Weierstrass, on peut montrer qu'il existe une sous-suite convergente de 
\( \left( \vb{x}_{k,2}^{(1)}\right)  \) pour certains indices $k$. De plus, la limite de cette suite, \( x_{1}^{*} \) est dans  \( [-\frac{1}{2^{2}},\frac{1}{2^{2}}]\)\\
Appelons cette sous-suite, \( \vb{x}_{k}^{(2)} \), puis définissions-là telle que 
\[
    \vb{x}_{k}^{(2)} = \vb{x}_{2k}^{(1)} = \vb{x}_{4k}^{(2)}.
\]
La suite définie en (\ref{suite1}) respecte également tous ces critères.\\\\
Il est à noter que, puisque la suite des premières composantes de \( \vb{x}_{k}^{(1)} \) est une 
sous-suite de \( \vb{x}_{k,1}^{(0)} \), qui converge, cette dernière converge également dans $[-\frac{1}{2},\frac{1}{2}]$.
\\\\
Finalement, on peut répéter le processus précédent pour la $m$-ième composante de chaque sous-suite précédente
\(
\vb{x}_{k,m}^{(m-1)},
\) en définissant \( \vb{x}_{k}^{(i)} \) telle que 
\[
\vb{x}_{k}^{(i)} = \vb{x}_{2^{i}k}^{(0)}. 
\]
, dont les $m$ premières composantes convergent vers une valeur \( x_{m}^{*} \in [-\frac{1}{2^{m}}, \frac{1}{2^{m}}] \). \\ 

Avec l'argument diagonal de Cantor, on peut donc choisir la sous-suite dont tous les éléments \(  \)
sur la "diagonale" \( m,k \) infinie dont tous les éléments convergent. Bien que le nombre de composantes soit infini,
on peut toujours construire une plus grande infinité de suites infinies afin d'en obtenir une convergente.

Pour terminer, puisque chaque composante $x_n$ de la sous-suite converge dans \( [-\frac{1}{2^{n}},\frac{1}{2^{n}}] \),
il est clair que cette sous-suite est dans \( K \). Chaque composante $x_n$ respecte 
\[
\lvert x_n \rvert \leq 2^{-n} \forall n in \mathbb{N}_{0}
\]
Il est également possible de remarquer que la suite définie en (\ref{suite1}) converge vers
\[
\vb{x} = (0,0,0,0,\ldots)
\]
%-----------------------------QUESTION 2 ----




%-----------------------------QUESTION 3 ----------------------------------%
\newpage
\section*{Question 3}


\begin{enumerate}[a)]

\item % a)

 % A REMPLIR PAR L'ETUDIANT:

\item % b)

 % A REMPLIR PAR L'ETUDIANT:

\item % c)

 % A REMPLIR PAR L'ETUDIANT:

\end{enumerate}


%-----------------------------QUESTION 4 ----------------------------------%
\newpage
\section*{Question 4}


\begin{enumerate}[a)]

\item % a)

 % A REMPLIR PAR L'ETUDIANT:

\item % b)

 % A REMPLIR PAR L'ETUDIANT:

\item % c)

 % A REMPLIR PAR L'ETUDIANT:

\end{enumerate}



\end{document}
