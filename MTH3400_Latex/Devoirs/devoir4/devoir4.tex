%--------------------------INITIALISATION DU DOCUMENT---------------------%
%											%														%											%
%                        >>>> Ne pas modifier cette partie <<<<			%
																																					
\documentclass[letterpaper,12pt,oneside,final]{book}



%%
%%  Version: 2014-10-28
%%
%%  Accepte les caractères accentués dans le document (UTF-8).
\usepackage[utf8]{inputenc}
%%
\usepackage{amsfonts}
%% Support pour l'anglais et le français (français par défaut).
%\usepackage[cyr]{aeguill}
\usepackage{lmodern}      % Police de caractères plus complète et généralement indistinguable visuellement de la police standard de LaTeX (Computer Modern).
\usepackage[T1]{fontenc}  % Bon encodage des caractères pour qu'Acrobat Reader reconnaisse les accents et les ligatures telles que ffi.
\usepackage[english,frenchb]{babel} % le langage par défaut est le dernier de la liste, c'est-à-dire français
%%
%% Charge le module d'affichage graphique.
\usepackage{graphicx}
\usepackage{epstopdf}  % Permet d'utiliser des .eps avec pdfLaTeX.
%%
%% Recherche des images dans les répertoires.
\graphicspath{{./images/}{./dia/}{./gnuplot/}}
%%
%% Un float peut apparaître seulement après sa définition, jamais avant.
\usepackage{flafter,placeins}
%%
%% Utilisation de natbib pour les citations et la bibliographie.
\usepackage{natbib}
%%
%% Autres packages.
\usepackage{amsmath,color,soulutf8,longtable,colortbl,setspace,ifthen,xspace,url,pdflscape,tikz,pgfplots}
%%
%% Support des acronymes.
\usepackage[nolist]{acronym}
\onehalfspacing                % Interligne 1.5.
%%
%% Définition d'un style de page avec seulement le numéro de page à
%% droite. On s'assure aussi que le style de page par défaut soit
%% d'afficher le numéro de page en haut à droite.
\usepackage{fancyhdr}
\fancypagestyle{pagenumber}{\fancyhf{}\fancyhead[R]{\thepage}}
\renewcommand\headrulewidth{0pt}
\makeatletter
\let\ps@plain=\ps@pagenumber
\makeatother
%%
%% Module qui permet la création des bookmarks dans un fichier PDF.
%\usepackage[dvipdfm]{hyperref}
\usepackage{hyperref}
\usepackage{caption}  % Hyperlien vers la figure plutôt que son titre.

\usepackage{esint}
\usepackage{geometry}
\usepackage{enumerate}




\begin{document}
%--------------------------------------------------------------------------------------%

%--------------------------PAGE DE COUVERTURE------------------------------%

% A REMPLIR PAR L'ETUDIANT: 

\newcommand\monPrenom{Frédéric}		%PRENOM
\newcommand\monNom{Laguë}			%NOM
\newcommand\monMatricule{1986131}	%MATRICULE
\newcommand\monGroupe{01}		%GROUPE

%------------------------ Ne pas modifier la ligne suivante --------------%
%\newgeometry{tmargin=2.0cm, bmargin=2.0cm, lmargin=2.25cm, rmargin=2.25cm, headsep=1.0cm}
\newgeometry{top=2cm}
\definecolor{gris1}{gray}{0.75}

\newcommand{\encadre}[1]{
\setlength\fboxsep{5mm}\setlength\fboxrule{1pt}
\begin{center}
\fcolorbox{black}{gris1}{
\begin{minipage}{0.94\textwidth}{#1}\end{minipage}}
\end{center}}

% encadre blanc
\newcommand{\boite}[1]{
\setlength\fboxsep{5mm}\setlength\fboxrule{1pt}
\begin{center}
\fcolorbox{black}{white}{
\begin{minipage}{0.5\textwidth}{#1}\end{minipage}}
\end{center}}


%\begin{document}

\thispagestyle{empty}

{
\centering

\encadre{
\begin{center}
\bf
{\Large \scshape 
Polytechnique Montr\'eal
\\
D\'epartement de Math\'ematiques et de G\'enie Industriel
}
\\
{\Huge
\

MTH3400 - Analyse math\'ematique pour ing\'enieurs
\\
Automne 2021

\

Devoir 3

}
\end{center}
}

\vfill

\fcolorbox{black}{white}{
\begin{minipage}{0.94\linewidth}

\vspace{5mm}

{\bf \Large Nom : }\monNom \hspace{20mm} {\bf \Large Pr\'enom : }\monPrenom%\rule[-1mm]{56mm}{0.6pt}

\vspace{8mm}

{\bf \Large Matricule : }\monMatricule \hspace{20mm} {\bf \Large Groupe : }\monGroupe

%\vspace{8mm}
%
%{\bf \Large Signature: }\rule[-1mm]{126mm}{0.6pt}
%
%

\end{minipage}}

\vspace{10mm}


{
\renewcommand{\arraystretch}{1.5}
\begin{center}
\begin{tabular}{|c|c|c|c||c|} \hline
{\bf \Large Q1}		& {\bf \Large Q2}	& {\bf \Large Q3}& 	{\bf \Large Q4} 	& {\bf \Large Total} \\ \hline
\hspace{20mm}	{\Huge \strut}	& \hspace{20mm}	& \hspace{20mm}	& \hspace{20mm}  & \hspace{20mm} \\ 
\hspace{20mm}	{\Huge \strut}	& \hspace{20mm}	& \hspace{20mm}	&  \hspace{20mm}  & \hspace{20mm} \\ \hline
\end{tabular}
\end{center}
}


\vfill

}

\restoregeometry
%\end{document}
%-------------------------------------------------------------------------%


%========================= Début des réponses ============================%


%-----------------------------QUESTION 1 ----------------------------------%
\section*{Question 1}


 % A REMPLIR PAR L'ETUDIANT:
Tout d'abord, pour montrer la linéarité de \( T{-1} \) on doit montrer que, \(\forall x,y \in U\):
\begin{enumerate}[i)]
    \item \[ T^{-1}(x+y) = T^{-1} x + T^{-1} y, \, \] et 
    \item \[ T^{-1}(\alpha x) = \alpha T^{-1}x \] 
\end{enumerate}

Pour i), posons \( x' = Tx, \, \in V  \text{ et } y' = Ty, \, \in V \). On peut donc écrire
\[ T ^{-1}(x' + y') = T^{-1}(Tx+Ty)  \]

Puisque \( T \) est linéaire, 
\[T^{-1}(Tx + Ty) = T^{-1}(T(x+y)) = T^{-1} T (x+y) = x+y  \] 

On avait \( x' = Tx\) et \( y' = Ty \). Puisque T est bijectif, \( x = T^{-1}x \) et \( y = T^{-1}y' \)
On trouve donc bel et bien
\[ T^{-1}(x' + y') = T^{-1}x' + T^{-1}y' \] 
\\ 

Pour ii), on pose \( x' = Tx  \). Similairement,

\[ T^{-1}(\alpha x') = T^{-1}(\alpha Tx)  \]

Puisque \( T  \) est linéaire, 
\[ T^{-1}(\alpha Tx ) = T^{-1}(T(\alpha x)) =  \alpha x\]
Puisque T est bijectif, \( x' = Tx \implies x = T^{-1} x \). 
On trouve donc effectivement
\[ T^{-1}(\alpha x) = \alpha T^{-1} (x)  \]

Donc \( T^{-1}  \) est donc linéaire. \\

Ensuite, pour la condition sur \( \lvert\lvert T^{-1} \rvert\rvert  \), on commence par écrire la définition de
la norme de \( T^{-1} \). 

\begin{equation*}
    \lvert\lvert T^{-1} \rvert\rvert   = \text{sup}\left\{ \frac{\lvert\lvert T^{-1}x \rvert\rvert _{U}}{\lvert\lvert x \rvert\rvert _{V}} |\, x\in V \text{ tel que }x\neq 0.  \right\}
\end{equation*}

Posons ensuite \( x= Tx',\, x\in V, \, x' \in U\). Puisque \( T \) est bijectif, on a \( x'= T^{-1}x\). On peut donc réécrire
\begin{equation*}
\begin{split}
\lvert \lvert  T^{-1} \rvert \rvert = \text{sup}\left\{ \frac{\lvert \lvert  x' \rvert \rvert _{U}}{\lvert \lvert  Tx' \rvert \rvert_{V}} |\, x' \in U \text{ tel que } x' \neq 0 \right\} 
\end{split}
\end{equation*}

En utilisant \( \lvert \lvert  Tx' \rvert \rvert _{V} \leq \lvert \lvert  T \rvert \rvert \ \cdot \lvert \lvert  x' \rvert \rvert _{U} \), (livre, p.53) on a que 
\[ \frac{1}{\lvert \lvert  T \rvert \rvert } \leq \frac{\lvert \lvert  x' \rvert \rvert _{U}}{\lvert \lvert  Tx' \rvert \rvert V} \implies \text{sup} \left\{\frac{\lvert \lvert  x' \rvert \rvert _{U}}{\lvert \lvert  Tx' \rvert \rvert V}\right\} \geq \frac{1}{\lvert \lvert  T \rvert \rvert }  \]

On trouve donc effectivement que
\[ \frac{1}{\lvert \lvert  T \rvert \rvert } \leq \lvert \lvert  T^{-1} \rvert \rvert  \]
%-----------------------------QUESTION 2 ----------------------------------%
\newpage
\section*{Question 2}

\begin{enumerate}[a)]

\item % a)
L'opérateur \( T \) est borné si: \( \exists M \in \mathbb{R}^{+}  \), tel que 
\[ \lvert \lvert  Tf \rvert \rvert \leq M\cdot \lvert \lvert  f \rvert \rvert _{\infty }, \, \forall f \in C^{0}[-a,a] \]
\[ \iff \lvert f(0) \rvert \leq M\cdot \sup_{x\in[-a,a]} \lvert f(x) \rvert , \, \forall f\in C^{0}[-a,a] \]

On remarque que pour \( M \) = 1, \( \lvert f(0) \rvert \leq \sup_{x\in [-a,a]} \lvert f(x) \rvert  \), si f n'est 
pas la fonction identiquement nulle.
L'opérateur est donc borné par \( \lvert \lvert  T \rvert \rvert \leq 1 \)

 % A REMPLIR PAR L'ETUDIANT:

\item % b)
Puisque \( T \) est un opérateur linéaire entre deux espaces vectoriels, \( T \) est continue si et 
seulement si il est borné. On doit donc montrer que \( T \) n'est pas borné. C'est-à-dire que, 
\( \forall M < \infty, \, \exists f \in C^{0}[-a,a] \) tel que
\[ \lvert f(0) \rvert > M \int_{-a}^{a} \lvert f(x) \rvert dx \]
Soit la fonction \( f_{n}(x)\) définie comme : 
\begin{equation*}
\lim_{n \to \infty}f_{n}(x) = \begin{cases}
        n^{2}x + n, &\;\; \text{ si } x\in [-\frac{1}{n},0] \\ 
        -n^{2}x +n &\;\; \text{ si } 0  x \in [0,\frac{1}{n}]\\ 
        0, & \;\; \text{ sinon}.
    \end{cases}
\end{equation*}
On remarque que \( \lim_{n \to \infty} f_n (0) = \lim_{n \to \infty} n = \infty \). De plus, on peut calculer que:
\begin{equation*}
\begin{split}
    lim_{n \to \infty} \int_{-a}^{a}\lvert f(x) \rvert dx & = \lim_{n \to \infty} \int_{-\frac{1}{n}}^{0} n^{2}x+ n\;dx + 
\int_{0}^{\frac{1}{n}} -n^{2}x + n\;dx \\ 
& = 1
\end{split}
\end{equation*}
Ainsi, \( \forall M < \infty, \; \lim_{n \to \infty} f_n(0) > M \lvert \lvert  f_{n} \rvert \rvert_{\infty } \)
et donc, on a montré que T n'est pas borné. 
Puisque T n'est pas borné, il n'est pas continu car il s'agit d'un opérateur linéaire, entre deux espaces vectoriels normés.
\end{enumerate}


%-----------------------------QUESTION 3 ----------------------------------%
\newpage
\section*{Question 3}

\begin{enumerate}[a)]

\item % a)
Pour montrer que \( \text{Ker}(T) \) est un sous-espace vectoriel on doit montrer 
\begin{enumerate}[i)]
    \item Fermeture sous l'addition:
\[ \forall \vb{x},\vb{y} \in \text{Ker}(T) \implies \; \vb{x}+\vb{y} \in \text{Ker}(T) \]
Pour prouver ceci, prenons \( \vb{x}, \vb{y} \in \text{Ker}(T)\). 
Alors, en utilisant la linéarité de \( T \):
\[ T(\vb{x}+\vb{y}) = T\vb{x} +T\vb{y} = \vb{0} + \vb{0} = \vb{0}\]
Donc \( \vb{x} + \vb{y} \) est dans le noyeau.
    \item Fermeture sous la multiplication par un scalaire (du corps \( \mathbb{F}\))
\[ \forall \alpha \in \mathbb{F}, \forall \vb{x} \in \text{Ker}(T) \implies alpba \vb{x} \in \text{Ker}(T)\]
Encore une fois, en utilisant le fait que \( T \) est linéaire: 
\[ T \alpha \vb{x} = \alpha T\vb{x} = \alpha \cdot \vb{0} = \vb{0} \]
Donc, \( \alpha \vb{x} \) est dans le noyeau. 
    \item Existence du vecteur nul dans le noyeau 
Pour prouver ceci, on fait ressortir l'élément inverse de l'addition de vecteurs, et en utilisant la linéarité. Soit \( \vb{x} \ in \text{Ker}(T) \)
\[ T\vb{0} = T(\vb{x+ \vb{-x}} = T(\vb{x}) + T(\vb{-x}) = T(\vb{x}) - T(\vb{x}) = \vb{0} - \vb{0} = \vb{0} \]
Le vecteur nul \( \vb{0} \)  est donc dans le noyeau.\\ 
\( \text{Ker}(T) \) est donc effectivement un sous-espace vectoriel. 
\end{enumerate}
 % A REMPLIR PAR L'ETUDIANT:

\item % b)
Pour montrer que \( \text{Ker}(T) \) est fermé, on utilise le fait que \( T \) est linéaire et borné, donc continu. 
Supposons que \( T \neq 0 \), et prenons \( \vb{x}_{n} \notin \text{Ker}(T) \implies \vb{x}_{n}\in \text{Ker}(T)^{C} \), défini comme:
\[ \vb{x_{n}} = \lim_{n \to \infty} \frac{\vb{x}}{n} | \vb{x} \notin \text{Ker}(T) \] 

Par la linéarité de \( T \), on peut voir que \( \vb{x}_{n} \) n'est pas dans le noyeau pour aucun \( n \);
\[ T(\vb{x}_{n}) = \lim_{n \to \infty} \frac{1}{n} T(\vb{x}) .\] 

Cependant, \( \vb{x}_{n} \) converge vers \( \vb{x'} = {0} \notin \text{Ker}(T)^{C} \). Cela signifie donc que la limite 
n'est pas dans le complément du noyeau. Le complément du noyeau est donc ouvert, et donc \( \text{Ker}(T) \) est fermé. 

 % A REMPLIR PAR L'ETUDIANT:

\item % c)
Tout d'abord, commençons par montrer que \( T \) est injectif \( \implies  \text{Ker}(T) = \left\{ \vb{0} \right\} \). 
Supposons que \( T \) est injectif. Par linéarité, 
\[ T(\vb{0}) = \vb{0} \implies \vb{0} \in \text{Ker}(T) \]
Ensuite, prenons \( \vb{x} \in U \).
On a donc, si \( \vb{x} \in \text{Ker}(T)\), que \( T(\vb{x}) = 0 \)\[ \implies T(\vb{x}) = T(\vb{0})\], puisque \( T \) est injectif, 
\[ \implies \vb{x} = \vb{0} \]
Ainsi, \( \vb{x} \in \text{Ker}(T) \implies \vb{x} = \vb{0}\) et donc 
\[ T \text{ est injectif } \implies \text{Ker}(T) = \left\{ \vb{0} \right\} \]
 % A REMPLIR PAR L'ETUDIANT:
Montrons ensuite que \( \text{Ker}(T) = \left\{ 0 \right\} \implies T \) est injectif. 
Soient \( \vb{x},\vb{y} \) tels que \( T\vb{x} = T\vb{y} \). Ainsi, 
\[ \implies T\vb{x}-T\vb{y} = 0 .\]
Par linéarité, 
\[ T\vb{x} - T\vb{y} = T(\vb{x}-\vb{y}) = 0 .\]
Donc, \( \vb{x}-\vb{y} \in \text{Ker}(T) \). 
Cependant, on avait \( \text{Ker}(T) = \left\{ \vb{0} \right\}  \)
\[ \implies \vb{x} - \vb{y} = \vb{0} \iff \vb{x} = \vb{y} \]
On a donc que \( T \) est injectif. 
Donc, \( \text{Ker}(T)=\left\{ 0 \right\} \implies T \) est injectif. 
\\ 

Puisque \( T \text{ est injectif } \implies \text{Ker}(T) = \left\{ \vb{0} \right\} \) et 
\( \text{Ker}(T)=\left\{ 0 \right\} \implies T \) est injectif, on a 
\[ \text{Ker}(T) = \left\{ \vb{0} \right\} \iff T \text{ est injectif} \]


\end{enumerate}


%-----------------------------QUESTION 4 ----------------------------------%
\newpage
\section*{Question 4}


\begin{enumerate}[a)]

\item % a)
Tout d'abord, on peut commencer par réécrire : 
\[ \lvert \lvert  A\vb{u} \rvert \rvert_{2}^{2} = \left( \sum_{k=1}^{n} (A\vb{u})_{k}^{2} \right)  \]
 % A REMPLIR PAR L'ETUDIANT:
En posant \( \vb{x} = A\vb{u} \), on obtient \( \lvert \lvert  \vb{x} \rvert \rvert_{2}^{2} = \sum_{k=1}^{n} \vb{x}_{k}^{2} \)
Ou, de façon équivalente 
\[ \lvert \lvert  \vb{x} \rvert \rvert^{2}_{2} = \vb{x}^{T}\vb{x} \]
Donc, 
\begin{equation*}
\begin{split}
    \lvert \lvert  A\vb{u} \rvert \rvert_{2}^{2}  & = (A\vb{u})^{T}A\vb{u} \\ 
& = \vb{u}^{T}A^{T}A\vb{u}
\end{split}
\end{equation*}

En posant \( F = A^{T}A \), on peut écrire en termes matriciels 
\begin{equation*}
\begin{split}
\lvert \lvert  A \vb{u} \rvert \rvert_{2}^{2} & = \sum_{i,j=1}^{n} u_{i}F_{ij}u_{j}\\
\end{split}
\end{equation*}
Donc le gradient est 
\begin{equation*}
\begin{split}
    \frac{\partial}{\partial u_{k}} \lvert \lvert A\vb{u} \rvert \rvert_{2}^{2} & = \sum_{j=1}^{n} F_{kj}u_{j} + \sum_{i=1}^{n} u_{i}F_{ik}  
\end{split}
\end{equation*}
Puisque \( F^{T} = (A^{T}A)^{T} = A^{T} A = F \)
On peut réécrire 
\[ \frac{\partial }{\partial u_{i}} \lvert \lvert  A\vb{u} \rvert \rvert_{2}^{2} = 2 \sum_{j=1}^{n} F_{ij}u_{j} = 2 A^{T}A\vb{u}\]

Le lagrangien s'écrit sous forme matricielle
\[\mathcal{L}(u_{i},\lambda) = \sum_{i,j = 1}^{n} u_{i}F_{ij}u_{j} + \lambda \sum_{i=1}^{n} (u_{i}^{2}-1) \]

Les dérivées sont donc, en utilisant le résultat précédent
\begin{equation*}
\begin{split}
\frac{\partial \mathcal{L}}{\partial u_{i}} = 2 \sum_{j=1}^{n} G_{ij}u_{j} - 2\lambda u_{i}
\end{split}
\end{equation*}
ou 
\[ \nabla_{\vb{u}} \mathcal{L} =  \]

Et
\[ \frac{\partial \mathcal{L}}{\partial \lambda} = \sum_{i=1}^{n} u_{i}^{2}-1 \]
\item % b)
On trouve les points critiques en posant 
\[ \frac{\partial \mathcal{L}}{\partial u_{i}}  = 0  \]
et 
\[  \frac{\partial \mathcal{L}}{\partial \lambda}  = 0 \]
Donc, 
\[ 2 \sum_{j=1}^{n} G_{ij}u_{j} - 2\lambda u_{i} = 0 \]
\[ \iff \sum_{j=1}^{n} G_{ij}u_{j} = \lambda u_{i} \]
\[ \iff G \vb{u} = A^{T}A \vb{u}= \lambda \vb{u} \]
Donc les points critiques correspondent à un vecteur propre de \( A^{T}A \). 
Puisque \( G \) est réelle et symétrique, elle est Hermitienne. On a montré en classe que les 
valeurs propres d'une matrice Hermitienne sont réelles.

\[ \vb{u^{*}} A^{T}A \vb{u} = \vb{u}^{*}\lambda \vb{u}  \]
\[ \iff (A\vb{u}^{T})A\vb{u} = \lambda \vb{u}^{T}\vb{u} \]
\[ \iff \lvert \lvert  A\vb{u} \rvert \rvert_{2}^{2} = \lambda \lvert \lvert  \vb{u} \rvert \rvert_{2}^{2} \]
 % A REMPLIR PAR L'ETUDIANT:
On doit donc avoir que \( \lambda \geq  0 \), car \( \lvert \lvert  A\vb{u} \rvert \rvert_{2}^{2}, \lvert \lvert  \vb{u} \rvert \rvert_{2}^{2} \geq 0 \)
On peut donc prendre la racine des deux côtés 
\[ \lvert \lvert A \rvert \rvert_{2} \lvert \lvert  \vb{u} \rvert \rvert_{2} \geq \lvert \lvert A\vb{u} \rvert \rvert _{2} = \sqrt{\lambda} \lvert \lvert  \vb{u} \rvert \rvert_{2}  \]
On trouve donc bien que \( \lvert \lvert  A \rvert \rvert_{2} \geq \sqrt{\lambda} \) où \( \lambda \) est une valeur propre réelle et positive. 

On a 
\begin{equation*}
\begin{split}
    \lvert \lvert  A \rvert \rvert_{2} & = \sup_{\lvert \lvert  \vb{u} \rvert \rvert_{2} = 1}   \left\{\lvert \lvert  A\vb{u} \rvert \rvert_{2}\right\}\\ 
& = \sup_{\lvert \lvert  \vb{u} \rvert \rvert_{2} = 1} \left\{\sqrt{\lambda}\lvert \lvert  \vb{u} \rvert \rvert_{2}\,|\, \text{il existe } \vb{u} \text{ tel que } A^{T}A\vb{u} = \lambda \vb{u}\right\}\\
& = \max\left\{\sqrt{\lambda}\,|\,\text{il existe } \vb{u} \text{ tel que } A^{T}A\vb{u} = \lambda \vb{u}\right\}\\ 
\end{split}
\end{equation*}
 % A REMPLIR PAR L'ETUDIANT:


 % A REMPLIR PAR L'ETUDIANT:

\end{enumerate}



\end{document}
